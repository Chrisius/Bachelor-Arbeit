\chapter*{Abstract}

By using currently popular tools like blogs or wikis and services like social networks, the users are becoming more and more involved into the production of content instead of just consuming it. Since those tools are familiar to the users, this can be an advantage because they do not have to get used to new tools as they can be embedded in a kind of framework. ... has the ability to integrate nearly every content that is available in the Internet. Another advantage of ... is the fact that the collaboration part and the social part of a ... is an already integrated functionality. 

Due to the ability of modern hand-held devices to determine its current position, the collaboration and networking factor can be extended by location-based services. It would be easier to communicate or to collaborate with other people if one knows where these persons are. People might spend less time for getting to know about the whereabouts of fellow students if they see who is around. Some situations can be handled more easily and effective if there is such a location-based PLE system that allows communication and collaboration. The implementation of an application for the ... Operating System will be focused on location-based scenarios for higher education. Students will have the possibility to offer help by providing skills and to receive help by searching for close-by people, to store and to retrieve information location-based, and to participate more active in lectures. This improves communication between students and helps them to improve their learning.
