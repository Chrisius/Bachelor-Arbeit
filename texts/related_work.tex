\chapter{Related Work}
\label{cha:related_work}

This chapter will talk about another approach of visualizing the provided data. Thomas Honné's project of his Master thesis ``Interactive Visualizations of Activity Patterns in Learning Environments'' \cite{thomasthesis} will be discussed and compared to the result of this thesis, \emph{SmartDay}.

Honné's \mnote{Visualizing} project and thesis describe a way of visualizing daily activities with a special focus to learning environments. Both project try to visualize the data in different ways, \emph{SmartDay} in a native way on Android and Honnè's project in a non native way with the help of a website. Since both visualizations need access to the Internet, \emph{SmartDay} is able to store data locally and display them even without an Internet connection. But the website based service also has its pros, because most platforms provide a browser which is capable of displaying Java Script, which is mainly used for the website.

Because both projects use data provided by the same gathering application \emph{BigBrother}, this comparison can focus on only visualization as both projects have the same precondition.

Both \mnote{The map} projects use different views to represent the provided data and both use a map for position data. Honné's map shows, in contrast to the visualization of \emph{SmartDay}, visited locations as unconnected circles. These circles differ in size and get larger, the longer one uses applications at this point. Those circles can consist of more than one position data, meaning that one can adjust a size in meter in which all data will be merged into one circle. This forms clusters on the map and creates personal points of interest, showing the user in an intuitive way, where he or she has used his or hers device the most. The map in \emph{SmartDay} connects each point and does not weight them by total amount of times applications have been used. This can be helpful to reconstruct routes but can also look messy sometimes.\\
Thomas Honné's map view also provides further details when clicking on the circle. The user is then displayed with a small windows which shows a pie chart displaying the applications used grouped by their productivity level. Furthermore, a bar chart,  similar to the timeline's detail view, is displayed, ordered descending by total time of usage. And at last one can see all events that were tracked at this position, listed as text in chronological order. \emph{SmartDay} may have pie charts too, but these are in a separated view, not position based and are not directly accessed by the map view. The details and completeness of the website's view dominates the map view presented in the application.

The \mnote{Visualizing of patterns} website provides a section not included in the native application, called \emph{Patterns}. This section makes use of Iurii Ignatko's work to recognize patterns with the help data mining in the provided dataset \cite{iuriisthesis}. It displays connections between application usages, for example it may show you the pattern, that if you use the calendar application, you are likely to use email client afterwards. This can be helpful especially to track down the root of usage patterns which distract one from his or hers work.

Another \mnote{Productivity and line charts} of the website is a tab called \emph{Productivity}. This view displays the same data as the detail window of the map tab, but takes adjustable timespans as input. This is a good solution with respect to productivity, as one can directly see his or hers daily, weekly or monthly usage of productive and non productive applications. The view \emph{Line Chart} presents the user a view which displays a line chart representing the number of events over time. The chart is freely adjustable, as one can add lines with restrictions to events and applications, colorize them and select a timespan for the it. It is an interesting feature which allows to compare different activities and applications. 

The website's views are reasoned and well structured. Especially the map view has some clear advantages over the application's view. The general concept and use of position based charts is a step ahead of the applications visualization. But the website still has some unused potential. For example, the productivity tab should take time, date and position into account, when weighting an application's productivity, as it may be okay to play a game at 22:00 at home on a Friday evening. In addition, the website lacks of a view, where the user can see his or hers day in a chronological order, like it is done in the applications timeline.