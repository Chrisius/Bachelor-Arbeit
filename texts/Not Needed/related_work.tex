\chapter{Related Work}
\label{cha:related_work}

With the spreading acceptance of mobile devices \cite{Mawston} featuring functionalities that are able to determine the position of a device, the amount of location-based applications rose. There are many use-cases for this kind of tools. In the following, six examples of applications are presented that are in a certain way similar to WaveLoc.


\section{Google Maps}

The service \emph{Google Maps}\footnote{http://www.google.com/mobile/maps/} is available for smart-phones like the iPhone, BlackBerry, or ... devices. Originally, it has been used as a digital map or for transit directions. Later on, several other features have been implemented, e.g. \emph{Latitude} that displays the current position of friends. There is also the possibility to find location-based information such as messages that have been send using \emph{Google Buzz}\footnote{http://www.google.com/buzz} (see figure \ref{fig:google_maps}). The position of the mobile device is assigned to that message -- if the user allows that -- so that it can be found by others who are searching for location-based information. There is also the possibility to add pictures to those messages. But it is not possible to directly talk to the senders of such messages or to get to know their current whereabouts without adding them to Latitude. One only sees a message, bound to a location. Another functionality of Google Maps is to display Points of Interest that are nearby, ordered by certain categories. In selected cities, it is also possible to get the schedules of public transportation \cite{googletransit}.

\ldots{}