\chapter{Background}
\label{cha:background}
In this chapter provides background information about the main topic of this bachelor thesis. First, the ongoing trend of using a smartphone in nearly every situation and thus need of keeping track of one's own mobile activities is discussed. To grant an application the possibility to give self reflecting impressions the application itself needs to be provided with personal data of the user. The second section talks about this provided data, its origin and how it is gathered. Once the origin of the information is discussed the next section talks about the representation of this it. In this context the meaning of native visualization is explained and an alternative is presented which involves a short introduction of a currently written master thesis. The last section lists and explains which hardware and software was used during the implementation.

\section{Self reflection}
Communication, social networks, camera, games, news, shopping, music and many more categories exists which describe a smartphone. As mentioned in the introduction, smartphones are used in nearly every situation. It is a modern multitool which is used by millions of people in Germany \cite{gstatistic}.\\
As \mnote{smartphones as modern multitools} advantageous as it may be in everyday situations, the downside is that most people do not know how much time they spend on their phone and thus do not know how distracting it may be.
For example, checking new mails may lead the user to also check the newest facebook messages and stay within this application a few minutes longer than expected. Because at least 50\% of smartphone owners use the Internet with their device \cite{gstatistic}, thus most people are always available through instant messaging services like whatsapp. The result is that people write and receive messages more often. And because smartphones are capable of running diverting games, one may be uses its device to beat the last high-score.

But a smartphone \mnote{Distraction} can be a great helper too. It is an easy to use digital calendar which reminds the user of all upcoming events, it can be used as a travel guide or a navigation system, it allows to quickly respond to an important email and has many more useful advantages. But the previous short examples demonstrate that a smartphone can also have a distracting influence to its owner. This influence may not be able to be assessed without the help of tools and background information, because keeping track of daily activities of a smartphone manually is nearly impossible.

At \mnote{Provide a self reflecting view} this point the help of tool is needed which provides information about the owners mobile day in a self reflecting manner. This tool in form of a smartphone application should display the information such that the user is able to instantly see where, when and for what the device was used. If one is provided with this data, he or she is able to say, that they used their smartphone in a productive manner or if they used it for entertainment. Further more, one could tell if he or she used the smartphone to divert themselves from working or studying. Also the application can only show the information, the conclusion must been drawn by the user. 

With this information a user of the application may rethink his or her work or study behavior, granting a boost in productivity and efficiency.


\newpage
\section{Provided Data}

\newpage
\section{Native Visualization}

\newpage
\section{Hardware and Software}