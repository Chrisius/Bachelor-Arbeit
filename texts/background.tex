\chapter{Background}
\label{cha:background}
This chapter provides background information about the main topic of this bachelor thesis. First, the ongoing trend of using a smartphone in nearly every situation and therefore the need of keeping track of one's own mobile activities is discussed. To grant an application the possibility to give self reflecting impressions the application itself needs to be provided with personal data of the user. The second section is about this provided data, its origin and how it is gathered. Once the origin of the information is discussed the next section talks about the representation of this it. In this context the meaning of native visualization is explained and an alternative is presented which involves a short introduction of a currently written Master thesis. The last section lists and explains which hardware and software was used during the implementation.

\section{Self reflection}
\textcolor{red}{ADD REFERENCE TO SELF REFLECTION PAPER!}\\
A Smartphone has many usabilities. It can be used as a camera, newspaper, music player or as a portable gaming device and those are just a few examples. There are a lot more ways to use a smartphone and as mentioned in the introduction, they are used in nearly every situation. It is a modern multitool which was used by more than 23 million people in Germany in 2012 \cite{gstatistic}.\\
As \mnote{Smartphones as modern multitools} advantageous as it may be in everyday situations, the downside is that most people do not know how much time they spend on their phone and thus do not know how distracting it may be.
For example, checking new mails may lead the user to also check the newest facebook messages and stay within this application a few minutes longer than expected. At least 50\% of smartphone owners access the Internet with their device \cite{gstatistic}, thus most people are always available through instant messaging services like whatsapp. The result is that people write and receive messages more often. And because smartphones are capable of running diverting games, one may uses its device to beat the last achieved high-score.

But a smartphone \mnote{Distraction} can be a great helper too. It is an easy to use digital calendar which reminds the user of all upcoming events, it can be used as a travel guide or a navigation system, it allows to quickly respond to an important email and has many more useful advantages. But the previous short examples demonstrate that a smartphone can also have a distracting influence to its owner.

At this point the the idea of self reflection is needed. The concept of self reflection is the critical reflection on one's own actions and positions and coming to a conclusion. This can be used to assess the distracting influence of smartphones to its owner. But for an accurate assessment one needs to keep track of his or hers own daily activities, which is nearly impossible to do for a smartphone without the help of tools that provide background information.

As \mnote{Provide a self reflecting view} mentioned, the help of a tool is needed which provides information about the owner's mobile day in a self reflecting manner. This tool in form of a smartphone application should display the information such that the user is able to instantly see where, when and for what the device was used. If one is provided with this data, he or she is able to say, that they used their smartphone in a productive manner or if they used it for entertainment. Further more, one could tell if he or she used the smartphone to divert themselves from working or studying. Although the application can display the needed information, the conclusion must been drawn by the user.

With \mnote{Possible improvements in productivity} this application and the provided information a user may be willing to rethink his or her work respectively study behavior. This then could lead to less frequent use in distracting applications, thus improving efficiency and productivity in daily tasks.

As described, there is a possible application area for such a smartphone application. It would visualize data and information about the owner's daily activities in such a way that the application could be used as a tool for self reflection.

\newpage
\section{Provided Data}
In the last section the idea of an application which displays information in a way such that one can use it for self reflection was described. What has not been described is the source of the data to be visualized and how it is gathered.

This \mnote{SmartDay and Big Brother} thesis' application, namely \emph{SmartDay}, will not gather the data it uses, instead the data is downloaded from a server and stored internally. The mentioned data arises from an external application called \emph{BigBrother}. This application is based on the Master thesis of Thorsten Kammer \cite{thorstensthesis} and was reimplemented and developed by Hendrik Th\"us in 2013 \cite{bigbrother}.\\
The data Big Brother gathers, is sent to a server where it can later be downloaded in an aggregated way and used by the application developed for this bachelor thesis. The data contains amongst others information about the user's visited locations, the name of the currently used application, start and end time of used applications. This data is uploaded and stored on a web server, which is then accessed by this thesis' application.

With \mnote{Privacy issues} the revelations published by Edward Snowden in June 2013 about the U.S. American spy program PRISM one might be concerned about privacy violation by third parties. It should just be said, that this project is still an experimental phase. If it should be published for a larger audience than the developers much work would be put into encryption and ensuring the prevention of unauthorized access by third parties. Another solution would be the release as an open source project. In this case users could host such a centralized service on their own.

One \mnote{Reasons for online stored data} of the reasons why the data of daily activities is stored online is the limited storage of mobile devices. This way the used size of storage can be minimized and only needed information can be downloaded. The possibility to merge data from other devices the user owns, like PCs, laptops and tablets is another reason to upload the data. Being able to access the data from multiple devices like tablets is also an important point. This method is of course at the cost of data traffic but is needed to centralize data especially in case of merging data and accessing them from different devices.

The idea of a data set which also contains information about other used devices has great potential in granting an even better overview of one's daily activities and thus this would make the self reflecting view provided by the application even more meaningful. Having the data of the activities of a laptop could be powerful for student who is learning with it, because he or she may not use the smartphone during that time, but instead one could see if he or she was productive, by checking if the laptop was actual used for studying by reading a pdf file or if it was used to browse non important things on the Internet. With additional data available from laptops or PCs, the coverage of one's day would be more complete.
%%%
\section{Native Visualization}
Now that the origin of the provided data has been explained, the idea of a native visualization will be explained in more detail.

The \mnote{What does visualization mean?} term of \emph{visualization} means representing of abstract data or information, like a text, in a visual ascertainable form. Its meaning is not limited to computer science. It can be found in various situations and places, technically anywhere where someone tries to convey information in a visual way. This may be a picture of an artist or a even a movie. The concept is not even limited to modern time. Since the early days of mankind, humans try to express information in form of visual tangible objects. For example the Egyptians did this 2000 years before Christ, by creating pectorals which for instance display a gryphon standing on a kneeling man of different skin color to express the Egyptians position in foreign countries \cite{sarahhallersthesis}.

As mentioned, in computer science, visualization means the representation of data in an illustrative way. For example, creating a pie chart for results of a survey or drawing a graph representing the daily temperatures for a week. For this thesis the visualization has to fulfills the task of displaying one's daily activities in an appealing and easily to understand way, such that one can directly draw conclusion from the information.

Native \mnote{Native visualization} visualization describes the creating of visual ascertainable objects only by means of resources that a specific system provides without any addition. A visualization would be the generation of a website with with the use of JavaScript and then displaying this website in the application. But this would be non native because of the use of a website to create the view.\\
In this thesis, the application will be implemented for the mobile operating system Android 4.0.3 and higher. Native visualization under Android means the use of Java and the access to the standard Android application programming interface. The application will not use JavaScript or anything else as this would infringe the terms of native visualization.

An \mnote{A non native visualization of daily activities} alternative to native visualization of activities can be found in Thomas Honné's Master thesis ``Interactive Visualizations of Activity Patterns in Learning Environments'' \cite{thomasthesis}. The thesis describes, among other topics, the visualization of daily activities in a web browser environment. For comparison an interesting fact is, that the data arises from the same application. For more information please refer to chapter \ref{cha:related_work}.

An advantage over the non native method is that data can easily be stored locally, thus making the application available for offline usage assuming that the needed data has been downloaded at some point in the past.\\
With a native application the user has tool for self reflection which is not permanently bound to an Internet connection and does not require any additional non native resources.
%%%%
\section{Hardware and Software}
As mentioned in the previous section the thesis' application will be developed for Android 4.0.3 and higher. To get an impression of the used tools in the development process this section describes the used software, the development environment and Android, as well as the used hardware, the development device.

The \mnote{The operating system of choice} application was developed for Android 4.0.3 ``Ice Cream Sandwich'' with the application programming interface level 15. To minimize overhead due to compatibility adjustments the support is only guaranteed for Android 4.0.3 and newer versions, which account for 63\% of all Android devices \cite{androidversionpercent}.\\
Android was the operating system of choice because it has a free developer license, great supportive community and with a share of 76.7\% in quart 2 of 2013 \cite{androidpercent}, Android is the largest market share holder of mobile operating systems.

The \mnote{The test device}testing device was the Motorola tablet Xoom with Android 4.0.3. The tablet with a screen size of 10.1 inches and resolution of 1280 times 800 pixel gave great advantage in the testing process of the application. Its screen is large enough to display all relevant data without minimizing their visual appearance. In addition the screen size allows precise testing of multi-touch gestures. The alternative to the real development device would have been an Android emulator provided by Google. With the emulator the development of this application would have been nearly impossible due to the fact that it does not support the ability to simulate multi-touch input with a normal mouse and the nonexistent support for GoogleMaps.

Google's \mnote{The development environment} recommended software development kit Eclipse with Android Development Tools served as development environment. Working and developing with Eclipse was simple and comfortable due to Google's numerous tutorials \cite{androidtutorials}. Neither installing the software development kit on a Windows 7 computer nor setting up new projects was a problem. %(After flipping the usb cable three times to get the correct side up, 
The test device connected to the computer was directly recognized by eclipse and executing and testing written code on the it proceeded without problems. Eclipse's debug mode was also very helpful at many points in the development process and helped to track down hidden bugs.

Working with Eclipse and Android was comfortable, a lot easier than expected and straightforward, even for a first time developer. Eclipse's improvement proposals and performance advices and Google's tutorials with helpful examples and background informations made this project a great educational experience.