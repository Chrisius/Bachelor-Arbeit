\chapter{Background}
\label{cha:background}

\chapterquote{The most compelling argument for the PLE is to develop educational technology which can respond to the way people are using technology for learning and which allows them to themselves shape their own learning spaces, to form and join communities and to create, consume, remix, and share material.}{Graham Attwell} 

This chapter provides background information on the main three topics on which this diploma thesis is based. First, the idea of a Personal Learning Environment is explained. It is shown how such a system can help the learner to organise, share, and publish the information. Afterwards, the functionalities of ... are explained. It is shown that ... can be used as a basis of a PLE by extending its functionalities. At the end of this chapter, a short introduction to the ... Operating System is given. This is a relatively new ... (released in 2008) which is utilised by mobile devices such as smart-phones or tablet PCs. The application WaveLoc that was implemented during this diploma thesis runs on ... because this is a promising \cite{canalysandroid} open source system and one can develop for it without being charged -- in contrast to developing for the iPhone \cite{iphonedev}.


\section{Personal Learning Environments}
\label{sec:background__personal_learning_environment}

\ldots{}