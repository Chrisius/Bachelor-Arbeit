\chapter{Introduction}
\label{cha:introduction}

%Since the Internet is available for everybody and is used commercially \cite{1629613}, it got more and more important. 
In \mnote{Mobile Internet becoming important} Europe, the number of the Internet users rose by 297.8\% between 2000 and 2009. In 2007, 56\% of the German population used the Internet on a daily basis \cite{statbundesamt} and in 2009, 73\% of the private households in Germany were able to access the Internet \cite{statba2010}. As of June 2010, 79\% of the German population are Internet users \cite{wius}. It is obvious that the Internet becomes more and more important. And if one takes a look at the rising amount of mobile devices that have been sold lately, one can say that \emph{mobile} applications using the Internet are becoming more and more important. In the second quarter of 2010, the global smart-phone shipments rose up by 43\% \cite{Mawston} while the shipment of ...based smart-phones rose up by 886\% in the second quarter of 2010 \cite{canalysandroid}. According to the \citet{bitkomhandy}, 10 million owners of cell-phones (including smart-phones) make use of the Internet capabilities of their devices (as of 2010). Applications for mobile devices are used regularly by 4 million smart-phone owners. This shows that there is a great potential in mobile devices, especially for those running the ... Operating System.

Social \mnote{Huge success for social networks} networks are a huge success concerning to their customers. According to the German StudiVZ, MeinVZ, and SchuelerVZ, they have over 16 million registered users \cite{userstudivz} and according to a Facebook press release, Facebook has 500 million active users worldwide of which 50\% log in on a daily basis \cite{userfacebook}. The people get more and more used to online networks and online tools, in which email, instant messaging, blogs and wikis are among the most popular services for communication and collaboration \citep{danic-introducing}. %Due to these tools and network structures, the users are no longer consumers without producing any content. They become themselves producers by sharing and creating content \cite{attwell-personal}.

\pagebreak
Considering \mnote{Social networks with location-based services} these two aspects -- the rising sales figures of smart-phones and the success of social networks -- there is a great potential in developing applications for mobile devices that enable their users to connect to each other. Due to the capabilities of modern mobile devices, the networking-factor can be extended by utilising the functionality of those devices to determine their current position. Thus, social networks with location-based services can be provided to enable real-time communication and collaboration.


\section{Objectives}
\label{sec:introduction__objectives}

\ldots{}