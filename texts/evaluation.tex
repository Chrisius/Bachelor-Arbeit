\chapter{Evaluation}
\label{cha:evaluation}
After finishing the implementation of SmartDay, a set of user tests have been conducted. The tests should unveil the strengths and problems of the application's usability and functionality.

As \mnote{Test users} test users, seven persons of age 21 to 46 have participated in the evaluation. Four students, two motorcar mechanics and one person with a job as secretary and paramedic have been asked various questions and should handle several tasks. The two women's and five men's school education reach from the German Berufsschulreife to the German Hochschulreife. All of the test users used a smartphone on a daily basis. Two mobile operating systems were present in this test run. One person used Windows Phone, two persons had experience with Windows Phone and Android and four users had an Android device. iOS was not used by any of the users.

The \mnote{Aim of the test} test was designed to show the users the most important parts of the graphical user interface. It should lead the testers from view to view and introduce them to the different functionalities. In order to introduce the usability of the application, tasks have been presented to the users which should be solved with the application's views and functionalities. Those tasks showed where the application could be improved with respect to functionality, overview and intuitiveness, but they also revealed the existing strengths of SmartDay.

The \mnote{The first interactions} test has been presented to each person in the same way, such that there were no considerable differences in the asked questions and testing environments. First, every person was told that he or she will be presented with an application visualizing his or hers fictional mobile data. The users were handed the Motorola Xoom as testing device and were requested to talk about all their thoughts, in order to catch their first impressions. After the short introduction, six of the seven persons have been given eight up to ten minutes in which they should test and discover the application. One person has used the application a few times during development and therefore did not need an introduction time. In this phase, the users were able to get a first impression of the views, options and functionalities.\\
The \mnote{MapView} six users looked at all views and tried out different options. The intention and use of the map view was recognizable for everyone. They used the map without problems and interacted with it by zooming. Some did not tap the displayed markers and from those who did, only one person tapped on the appearing speech bubble to switch the view. The map view's option \emph{highlight apps} was used by three persons, but not all of them recognized the new coloration. The desire for more structure by merging markers which are close together and visualizing the temporal sequence of the visited locations have been mentioned.\\
The \mnote{ChartView} chart view seemed to be the most non intuitive view in the starting phase. Most people did not tap on the pie slices and thus were not able to recognize the changing details on the right side of the layout. Even though most people were restrained when tapping on the views, all pressed the text ``show location'' in the details at least once, but none was considering pressing on one of the usage timespans, this even holds for the highlighted timespans in the details of the timeline.\\
The \mnote{Timeline} timeline was used by most testers without greater problems. Most of the users tapped on the rectangles to unveil the details and zoomed in and out. The highlighting of selected application timespans was not recognized in the timeline but has been seen in the details. Even though the users tapped on the application names, none tapped on a specific timespan in the details. One person also asked for a total time of usage, as the application bars in the details represent the percentage of usage time. Zooming in and out by double tapping was not discovered by the users, although at least one person has used this gesture in the map view.\\
Interaction \mnote{Usage of options} with options have rarely been observed. But the selection of a new timespan or a new day has been used by every person, although two testers did not immediately anticipate, that they are able to tap on the shown days to select a new timespan. Colorization of applications have been used by two testers, but \emph{Apps} and the three dots on the top right corner, which reveal not visible options have not been tapped. Two persons also wanted to use the back button on the tablets bottom to return to the last view and closed the application this way.\\
Although not all features have been found immediately, all test users were able to interact with the application and understood the intention of the shown views.

After the introduction time the test users have been confronted with tasks and questions which can be found in the appendix \ref{cha:appendixEva}.\\
The \mnote{Manual detection of locations} users were first asked to determine their fictional home and work place. All were able to identify these places with different strategies. The quickest way was to solve this was the combined use of timeline and map view. One person selected an application used at a time where one is normally at home, for example 8:00 and clicked the respective ``show location'' button. Another solution was the identification of areas with a high marker density, but was not as quick as the first solution, because the users had to choose between different areas by comparing the times of the day. This solution could also be inaccurate as the marker's information only shows the used applications but not their used times. If one uses many applications for a few seconds at the bus station and only one or two for a long time at home, one could confuse both locations. This could be prevented by displaying markers in different sizes, depending on the time of application usages.\\
Five \mnote{Determination of distracting influence} of the seven participants could answer the question, if they have been distracted from work by their mobile device. They used the timeline and observed the timespan from 9:00 to 17:00 and found nearly no activity during this time. The two users who could not answer the question the timeline did not provide enough information for them to draw this conclusion. The result may have been different, if the timeline would have provided information about position above the rectangles.\\
Questions \mnote{Productivity} about the primary use of the smartphone could be answered with certainty. All test users chose the chart view over the detail view of the timeline to answer this question. With the ability to group applications, this task could have been speed up, because one would not need to identify each application by name but instead could just focus three major groups - productive, neutral and not productive. This grouping has also been stated as a missed feature by one person in the first stage of the test.\\
The last question concerning self reflective properties was the determination of patterns in daily activities. All persons were able to draw conclusions with the use of the timeline to identify patterns. The morning routine and work breaks were found very quickly, allowing the users to make statements about one's daily activities. All participants mainly used the full timeline form 0:00 to 24:00 for every day to identify daily patterns by comparing the occurring colors. One user stated the wish for overlaying all days to see which colors dominate which parts of the day. This might be helpful to track down distracting patterns and help boost to productivity.

The \mnote{Interacting with SmartDay} test users have also been provided with tasks regarding the use of the applications interface. Those tasks concerned for example the coloring of applications and showing or hiding of applications. It unveiled that all users were focused on the main view and wanted to apply changes directly through the views. For instance, four users tried to change the color of an application double tapping or tap and hold on the applications name, although the option was presented next to the tab headers.\\
Some features, for instance the tapping on a speech bubble at the map view, the selection of a slice of the pie chart and the selection of timespan in detail view of the timeline have not been used by most users. Reasons may be a non intuitive visualization or the lacking of experience with the application, because all users were cautious when testing the application in the first phase and mainly focused on understanding the views themselves.

The \mnote{Stated critics} last stage of the test was the filling of a form, asking for critics and suggestions. Two users stated that they would like to have a German version of the application. The improvement of the map section was also suggest, because the view looses its overview if more days are selected and the merging of markers lying close together was also suggested. Concerns about the data's safety were also stated as some users where unsure what will happen with their tracked information beside the visualization in SmartDay.

This evaluation showed, that the application SmartDay is a suitable tool for self reflection. The test users were able to identify points of interest and patterns in daily activities. Furthermore did the presented views allow the user to identify patterns in one's daily activities. The tests did also show that the application has room for improvements and enhancements, for example the implementation of grouping of applications and the merging of position markers to strengthen the clarity of views.