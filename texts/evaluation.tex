\chapter{Evaluation}
\label{cha:evaluation}

To \mnote{Amount of test-persons} test and to improve the usability of WaveLoc, three sets of user-tests have been conducted. The first two with five users, the last one with three users. According to \citet{ruleof5}, it is sufficient to conduct user-tests with five users. Thus, 17 user-tests with different students (including the final set of tests (see chapter \ref{sec:evaluation__final_tests})) are more than required. As stated, the number of usability problems that are found in a test with $n$ users and a total number of $N$ problems is

$$N(1-(1-L)^n)$$

$L$ \mnote{Detecting most of the problems} denotes the proportion of usability problems discovered while testing a single user with a typical value of 31\%. According to this formula, about 84\% of the $N$ problems can be found if five users are tested. The second set of user-tests with five persons should reveal another 13\% of the original $N$ problems. Thus, conducting user-tests with 10 persons should result in finding about 97-98\% of the problems concerning the usability.

After conducting these three user-tests, a final test with 4 users has been conducted. The reason for this set was again to test the changes that have been implemented after each set of the first user-tests. The participants should consider the application operable intuitively.



The first two sets of user-tests have each been conducted with five students while the third one has been conducted with three students, each of them having different foreknowledge concerning smart phones.

\ldots{}