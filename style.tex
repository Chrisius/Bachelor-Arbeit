% Leichte Variation von http://hci.rwth-aachen.de/karrer_thesistemplate
% Weiter leichte veränderungen für Mo - v=0.9b
% =============
% = Changelog =
% =============
% [0.99b]
% new - DotVarDescription environment: Adds dots between label and description.
% new - Addes some listings styles from Max Speicher (@maxspeicher) for the listings package and some additional from Gregor Aisch.
% changed - VarDescription: The description label is not aligned to the left.
%
% [0.98b]
% changed - styling of headings (section to subparagraph) from titleformat to addtokomafont so it is
%           possible to change the style with \addtokomafont{\Large} etc. from the main document.
%
% [0.97b]
% new - abstract environment -> \begin{abstract}[name (default none)] TEXT \end{abtract}
% new - acknowledgments environment -> \begin{acknowledgments}[Acknowledgments] TEXT \end{acknowledgments}
%
% [0.96b]
% new - compatibility with scrbook \part command (sectsty package is incompatible)
% removed - sectsty package (replaced functionality with titlesec commands)
%
% [0.953b]
% new - added hyphen option to url-package to allow line breaks after hyphens "-" in urls
%
% [0.952b]
% changed - automatically set header to 'Bibliography' with \printbibliography
%
% [0.951b]
% new - bibliography heading changable with \def\bibheadingonline{TEXT} and \def\bibheadingoffline{TEXT}
% changed - \mnote color (light blue) and \mnote fontsize normal changeable to footnotesize with \def\mnotefootnotesize{}
% changed - Seperate bibliography titles - references -> References
%
% [0.95b]
% new - changed to biblatex
% new - command \setwidesite to be called before \printbibliography to increase side width
% new - biblatex filter online (@online) and offline (!@online)
% dep - \widebibliography
%
% [0.9b]
% new - \usepackage{soul} for higlighting (\hl{some text} etc.)
% new - \shl{some text} to mark text red, no background marking
%
% [0.85b]
% new - Environment VarDiscription | \begin{VarDescription}{widest word}
% new - \widebibliography{} command
% new - scenario list (alph label and only 2pt item seperation)
% changed - changed seperator between multiple citations from ',' to ';'
% changed - changed labels in itemize list
% changed - csquote style to american (starts quotations with double quote instead of single quotes (inverted commas)).
% changed - csquote cite command to use natbibs \citep command.
%
% [0.8b]
% new - \footcite[]{} command
% new - Changed color of item/enum and description label to LightBlue



\usepackage[hyphens]{url}

%
%   B I B L I O G R A P H Y   &   C I T A T I O N S
%
%\usepackage[square]{natbib}  % round?
% \usepackage[round]{dinat}

% \usepackage{bibgerm}
% \usepackage{natbib}
% \bibliographystyle{plainnat}
% \bibliographystyle{gerapali}
% \bibliographystyle{natdin}
% \bibliographystyle{natdineng}
% \bibpunct{(}{)}{;}{a}{,}{,}
% \bibpunct{[}{]}{,}{a}{,}{,}
% \bibpunct{[}{]}{;}{a}{,}{,}
% \newcommand{\fullcite}{\cite}
% \newcommand{\footcite}[2][]{\footnote{#1 \citet{#2}}}

% ============
% = BibLaTeX =
% ============
% natbib for compability with natbib style citing and authoryear type
\usepackage[style=numeric,natbib=true]{biblatex}
% \usepackage[style=alphabetic,natbib=true]{biblatex}
% \usepackage[style=authoryear,natbib=true]{biblatex}
% \usepackage[style=authoryear,natbib=true, useprefix=true]{biblatex}

\renewcommand{\bibsetup}{%
  \markboth{\MakeUppercase{Bibliography}}{}
}

\newcommand{\setwidesite}{%
  \fancyhfoffset[LE,RO]{\marginparsep+\marginparwidth-4cm}
  \fancyheadoffset[LO,RE]{0cm}
  \fancyfootoffset[RE]{2cm}
  \newgeometry{inner=2cm, outer=2cm, bottom=4cm}}

\ifdefined\bibheadingonline
  \defbibheading{online}{\section*{\bibheadingonline}}
\else
  \defbibheading{online}{\section*{Online References}}
\fi
\ifdefined\bibheadingoffline
  \defbibheading{offline}{\section*{\bibheadingoffline}}
\else
  \defbibheading{offline}{\section*{Printed References}}
\fi

\defbibfilter{online}{%
  \( \type{online} \)}

\defbibfilter{offline}{%
  \( \not \type{online} \)}
  
% \DeclareNameFormat{sortname}{%
%   \iffirstinits
%   {\usebibmacro{name:last-first}{#1}{#4}{#5}{#7}}
%   {\usebibmacro{name:last-first}{#1}{#3}{#5}{#7}}%
%   \usebibmacro{name:andothers}}
  




%
% LANGUAGE & ENCODING
%
\usepackage[english]{babel}
% \usepackage[ngerman]{babel}
% Bug: this statement seems to be useless, instead, the _last_
% language from the parameterlist to \usepackage{babel} seems to be
% important
% \selectlanguage{ngerman}
\selectlanguage{english}
\usepackage[T1]{fontenc}
% \usepackage[latin1]{inputenc}   % can use native umlauts
\usepackage[utf8]{inputenc}   % can use native umlauts
% \usepackage[babel,german=quotes]{csquotes}  % provides \enquote{Blupp} => "`Blupp"'
\usepackage[babel,english=american]{csquotes}	% provides \enquote{Blupp} => "`Blupp"'
% \SetCiteCommand{\citep}
\SetCiteCommand{\parencite} % Changed for biblatex

\usepackage{units}              % unified way of setting values with units

\usepackage{appendix}
% \renewcommand{\appendixpagename}{Anhänge}
% \renewcommand{\appendixtocname}{Anh"ange}
%
% FONTS
%
% \usepackage{times}              % use times font
% \usepackage{lmodern}            % use lmodern fonts
%change standard fonts to Palatino and Helvetica
% \usepackage{palatino}
\usepackage{mathpazo} 
\usepackage[scaled=.95]{helvet} 
\usepackage{courier} 

\usepackage{noindent} %do not indent at new paragraphs but add a vertical offset

\usepackage{url}

%
% SYMBOLS
%
% \usepackage[right]{eurosym}
\usepackage{gensymb}
\usepackage{textcomp}           % for \textmu (non-italic $\mu$)
% \usepackage{marvosym}
\makeatletter					% make "@" a regular letter, not a special symbol


%\setcounter{secnumdepth}{3}     % limit enumeration depth
%\setcounter{tocdepth}{3}        % limit TOC depth

% tables & table formatting
\usepackage{tabularx}
\usepackage{booktabs}
\usepackage{multirow}
% \usepackage{dcolumn}%D{,}{,}{Nachkommastellen}

% \setlength{\itemsep}{5mm plus10mm minus20mm}


\fboxsep0mm

\usepackage{amsmath}            % math fonts
\usepackage{amssymb}            % math symbols
\usepackage{setspace}           % line spacing 

% package for higlighting etc. (\hl{}...)
\usepackage{soul}

% simple highlight without soul package
\newcommand{\shl}[1]{\textcolor{red}{#1}}

%
% C O D E   L I S T I N G S
%

%\usepackage{fancyvrb}           % algorithm-boxes
\usepackage{listings}

% \lstset{
%   numbers=left,
%   numberstyle=\tiny,
%   numbersep=5pt,
%   breaklines=true,
%   stepnumber=5,
%   tabsize=2,
%   basicstyle=\ttfamily\small,
%   frame=none,
%   numberfirstline=true,
%   firstnumber=1
%   }
% ============================
% = Listings Styles from Max =
% ============================

\definecolor{violet}{cmyk}{0.45,0.97,0.27,0.21}
\definecolor{lstblue}{cmyk}{1,0.80,0,0}
\definecolor{lstgreen}{cmyk}{0.71,0.21,0.65,0.22}
\definecolor{bluegrey}{cmyk}{0.56,0.24,0.11,0.05}
\definecolor{javadoc}{cmyk}{0.88,0.59,0,0}
\definecolor{lstgrey}{cmyk}{0.55,0.44,0.42,0.32}

\lstdefinelanguage{SQL}{
     keywords={},
     keywordstyle=\color{bluegrey}\bfseries,
     morekeywords=[2]{CREATE,TABLE,IF,NOT,EXISTS,NULL,SET,DEFAULT,PRIMARY,KEY,COLLATE,CHARACTER,AUTO_INCREMENT,ENGINE,CHARSET},
     keywordstyle={[2]\color{violet}\bfseries},
     otherkeywords={int,varchar,double,text,tinyint},
     sensitive=false,
     morecomment=[l][\color{lstgreen}]{//},
     morecomment=[s][\color{lstgreen}]{/*}{*/},
     morecomment=[s][\color{javadoc}]{/**}{*/},
     morestring=[b]',
     morestring=[b]"
  }
\lstdefinelanguage{PHP}{
     keywords={},
     keywordstyle=\color{bluegrey}\bfseries,
     morekeywords=[2]{static,function,if,return,pow,sin,cos,asin,min,sqrt,int},
     keywordstyle={[2]\color{violet}\bfseries},
     otherkeywords={@param, @returns, @author, @type, @link, @see},
     sensitive=false,
     morecomment=[l][\color{lstgreen}]{//},
     morecomment=[s][\color{lstgreen}]{/*}{*/},
     morecomment=[s][\color{javadoc}]{/**}{*/},
     morestring=[b]',
     morestring=[b]"
  }
\lstdefinelanguage{JavaScript}{
     keywords={},
     keywordstyle=\color{bluegrey}\bfseries,
     morekeywords=[2]{attributes, class, classend, do, empty, endif, endwhile, fail, function, functionend, if, implements, in, inherit, inout, not, of, operations, out, return, set, then, types, while, use},
     keywordstyle={[2]\color{violet}\bfseries},
     otherkeywords={@param, @returns, @author, @type, @link, @see},
     sensitive=false,
     morecomment=[l][\color{lstgreen}]{//},
     morecomment=[s][\color{lstgreen}]{/*}{*/},
     morecomment=[s][\color{javadoc}]{/**}{*/},
     morestring=[b]',
     morestring=[b]"
  }
\lstdefinelanguage{Java}{
     keywords={},
     keywordstyle=\color{bluegrey}\bfseries,
     morekeywords=[2]{abstract,boolean,break,byte,case,catch,char,class,
      const,continue,default,do,double,else,extends,false,final,
      finally,float,for,goto,if,implements,import,instanceof,int,
      interface,label,long,native,new,null,package,private,protected,
      public,return,short,static,super,switch,synchronized,this,throw,
      throws,transient,true,try,void,volatile,while},
     keywordstyle={[2]\color{violet}\bfseries},
     morekeywords=[3]{@SuppressWarnings, @Capability, @Override},
     keywordstyle={[3]\color{lstgrey}},
     otherkeywords={@param, @return, @returns, @author, @link, @see},
     sensitive,
     morecomment=[l]//,
     morecomment=[s]{/*}{*/},
     morecomment=[s][\color{javadoc}]{/**}{*/},
     morestring=[b]",
     morestring=[b]',
  }[keywords,comments,strings]

% some listings styles from Gregor Aisch
% http://vis4.net/blog/2009/09/noch-mehr-sprach-definitionen-fuer-latex-listings/

\lstdefinelanguage{HTML5} {morekeywords={a, abbr, address, area, article, aside, audio, b, base, bb, bdo, blockquote,  body, br, button, canvas, caption, cite, code, col, colgroup, command, datagrid, datalist, dd, del, details, dialog, dfn, div, dl, dt, em, embed, eventsource, fieldset, figure, footer,  form,  h1, h2,  h3,  h4, h5,  h6,  head,  header,  hr, html,  i, iframe,  img,  input,  ins, kbd,  label,  legend,  li,  link,  mark,  map,  menu,  meta,  meter,  nav,  noscript,  object,  ol,  optgroup,  option,  output,  p,  param,  pre,  progress,  q,  ruby,  rp,  rt,  samp,  script,  section,  select,  small,  source,  span,  strong,  style,  sub,  sup,  table,  tbody,  td,  textarea,  tfoot,  th,  thead,  time,  title,  tr,  ul,  var,  video},
sensitive=false, morecomment=[s]{<!--}{-->}, morestring=[b]", morestring=[d]'}

\lstdefinelanguage{CSS} {morekeywords={azimuth,  background-attachment,  background-color,  background-image,  background-position,  background-repeat,  background,  border-collapse,  border-color,  border-spacing,  border-style,  border-top, border-right, border-bottom, border-left,  border-top-color, border-right-color, border-bottom-color, border-left-color,  border-top-style, border-right-style, border-bottom-style, border-left-style,  border-top-width, border-right-width, border-bottom-width, border-left-width,  border-width,  border,  bottom,  caption-side,  clear,  clip,  color,  content,  counter-increment,  counter-reset,  cue-after,  cue-before,  cue,  cursor,  direction,  display,  elevation,  empty-cells,  float,  font-family,  font-size,  font-style,  font-variant,  font-weight,  font,  height,  left,  letter-spacing,  line-height,  list-style-image,  list-style-position,  list-style-type,  list-style,  margin-right, margin-left,  margin-top, margin-bottom,  margin,  max-height,  max-width,  min-height,  min-width,  orphans,  outline-color,  outline-style,  outline-width,  outline,  overflow,  padding-top, padding-right, padding-bottom, padding-left,  padding,  page-break-after,  page-break-before,  page-break-inside,  pause-after,  pause-before,  pause,  pitch-range,  pitch,  play-during,  position,  quotes,  richness,  right,  speak-header,  speak-numeral,  speak-punctuation,  speak,  speech-rate,  stress,  table-layout,  text-align,  text-decoration,  text-indent,  text-transform,  top,  unicode-bidi,  vertical-align,  visibility,  voice-family,  volume,  white-space,  widows,  width,  word-spacing,  z-index},
sensitive=false, morecomment=[s]{/*}{*/}, morestring=[b]", morestring=[d]'}

\lstdefinelanguage{JavaFX} {morekeywords={abstract, after, and, as, assert, at, attribute, before, bind, bound, break, catch, class, continue, def, delete, else, exclusive, extends, false, finally, first, for, from, function, if, import, indexof, in, init, insert, instanceof, into, inverse, last, lazy, mixin, mod, new, not, null, on, or, override, package, postinit, private, protected, public-init, public, public-read, replace, return, reverse, sizeof, static, step, super, then, this, throw, trigger, true, try, tween, typeof, var, where, while, with },
sensitive=false, morecomment=[l]{//}, morecomment=[s]{/*}{*/}, morestring=[b]", morestring=[d]'}

\lstdefinelanguage{MXML} {morekeywords={mx:Accordion, mx:Box, mx:Canvas, mx:ControlBar, mx:DividedBox, mx:Form, mx:FormHeading, mx:FormItem, mx:Grid, mx:GridItem, mx:GridRow, mx:HBox, mx:HDividedBox, mx:LinkBar, mx:Panel, mx:TabBar, mx:TabNavigator, mx:Tile, mx:TitleWindow, mx:VBox, mx:VDividedBox, mx:ViewStack, mx:Button, mx:CheckBox, mx:ComboBase, mx:ComboBox, mx:DataGrid, mx:DateChooser, mx:DateField, mx:HRule, mx:Image, mx:Label, mx:Link, mx:List, mx:Loader, mx:MediaController, mx:MediaDisplay, mx:MediaPlayback, mx:MenuBar, mx:NumericStepper, mx:ProgressBar, mx:RadioButton, mx:RadioButtonGroup, mx:Spacer, mx:Text, mx:TextArea, mx:TextInput, mx:Tree, mx:VRule, mx:VScrollBar, mx:Application, mx:Repeater, mx:UIComponent, mx:UIObject, mx:View, mx:FlexExtension, mx:UIComponentExtension, mx:UIObjectExtension, mx:Fade, mx:Move, mx:Parallel, mx:Pause, mx:Resize, mx:Sequence, mx:WipeDown, mx:WipeLeft, mx:WipeRight, mx:WipeUp, mx:Zoom, mx:EventDispatcher, mx:LowLevelEvents, mx:UIEventDispatcher, mx:CurrencyFormatter, mx:DateFormatter, mx:NumberFormatter, mx:PhoneFormatter, mx:ZipCodeFormatter, mx:CursorManager, mx:DepthManager, mx:DragManager, mx:FocusManager, mx:HistoryManager, mx:LayoutManager, mx:OverlappedWindows, mx:PopUpManager, mx:SystemManager, mx:TooltipManager, mx:CreditCardValidator, mx:DateValidator, mx:EmailValidator, mx:NumberValidator, mx:PhoneNumberValidator, mx:SocialSecurityValidator, mx:StringValidator, mx:ZipCodeValidator, mx:DownloadProgressBar, mx:ArrayUtil, mx:ClassUtil, mx:Delegate, mx:ObjectCopy, mx:URLUtil, mx:XMLUtil, mx:CSSSetStyle, mx:CSSStyleDeclaration, mx:CSSTextStyles, mx:StyleManager, mx:HTTPService, mx:RemoteObject, mx:Service},
sensitive=false, morecomment=[s]{<!--}{-->}, morestring=[b]", morestring=[d]'}

\lstdefinelanguage{LZX} {morekeywords={a, alert, animator, animatorgroup , attribute, audio , axis, axisstyle , b, barchart, basebutton , basebuttonrepeater , basecombobox , basecomponent , basedatacombobox , basedatepicker , basedatepickerday , basedatepickerweek , basefloatinglist , basefocusview , baseform , baseformitem , basegrid , basegridcolumn , baselist , baselistitem , basescrollarrow , basescrollbar , basescrollthumb , basescrolltrack , baseslider , basestyle , basetab , basetabelement , basetabpane , basetabs , basetabsbar , basetabscontent , basetabslider , basetrackgroup , basetree , basevaluecomponent , basewindow , br , button , canvas , chart , chartbgstyle , chartstyle , checkbox , class , columnchart , combobox , command , connection , connectiondatasource , constantboundslayout , constantlayout , datacolumn , datacombobox , datalabel , datamarker , datapath , datapointer , dataselectionmanager , dataseries , dataset , datasource , datastyle , datastylelist , datatip , datepicker , debug , dragstate , drawview , edittext , event , face , floatinglist , font , font , form , frame , grid , gridcolumn , gridtext , handler , hbox , horizontalaxis , hscrollbar , i , image , img , import , include , inputtext , javarpc , label , labelstyle , layout , legend , library , linechart , linestyle , list , listitem , LzTextFormat , menu , menubar , menuitem , menuseparator , method , modaldialog , multistatebutton , node , p , param , piechart , piechartplotarea , plainfloatinglist , plotstyle , pointstyle , pre , radiobutton , radiogroup , rectangularchart , regionstyle , remotecall , resizelayout , resizestate , resource , reverselayout , richinputtext , rpc , script , scrollbar , security , selectionmanager , sessionrpc , simpleboundslayout , simpleinputtext , simplelayout , slider , soap , splash , stableborderlayout , state , statictext , style , submit , swatchview , SyncTester , tab , tabelement , tabpane , tabs , tabsbar , tabscontent , tabslider , Test , TestCase , TestResult , TestSuite , text , textlistitem , tickstyle , tree , u , valueline , valuelinestyle , valuepoints , valuepointstyle , valueregion , valueregionstyle , vbox , verticalaxis , view , view , vscrollbar , webapprpc , window , windowpanel , wrappinglayout , XMLHttpRequest , xmlrpc , zoomarea},
sensitive=false, morecomment=[s]{<!--}{-->}, morestring=[b]", morestring=[d]'}

\lstset{
	numbers=left,
	numberstyle=\tiny,
	numbersep=5pt,
	breaklines=true,
	stepnumber=5,
	tabsize=2,
	basicstyle=\ttfamily\small,
	frame=none,
	numberfirstline=true,
	firstnumber=1,
	keywordstyle=\color{violet}\bfseries,
	ndkeywordstyle=\color{bluegrey}\bfseries,
	identifierstyle=\color{black},
  commentstyle=\color{lstgreen}\ttfamily,
  stringstyle=\color{lstblue}\ttfamily,
  showstringspaces=false
	}

	
% frame = none, single, topline,bottomline
% columns = fixed

% ===========================
% = Change list definitions =
% ===========================

% change color of item list
\renewcommand{\labelitemi}{\color{LightBlue}$\bullet$}
\renewcommand{\labelitemii}{\color{LightBlue}$\circ$}
\renewcommand{\labelitemiii}{\color{LightBlue}$\ast$}
\renewcommand{\labelitemiv}{\color{LightBlue}$\diamond$}

% change color of enum list
\renewcommand{\labelenumi}{\color{LightBlue}\arabic{enumi}.}
\renewcommand{\labelenumii}{\color{LightBlue}\alph{enumii})}
\renewcommand{\labelenumiii}{\color{LightBlue}\roman{enumiii}.}
\renewcommand{\labelenumiv}{\color{LightBlue}\Alph{enumiv}.}

% change color of description list
\usepackage{enumitem}
\setdescription{font=\color{LightBlue}\rm\itshape}
% \renewenvironment{description}{\list{font=\color{LightBlue}\itshape}}{\endlist}

% change color of footnotes
\renewcommand{\thefootnote}{\color{LightBlue}\arabic{footnote}}


% use nice footnote indentation
\deffootnote[1em]{1em}{1em}{\textsuperscript{\thefootnotemark}\,}


%
%	G R A P H I C S   A N D   I M A G E S
%
\usepackage{graphicx}
\graphicspath{{images/}}		% path to your image folder

\usepackage{eso-pic}	% needed for the full-face titlepage
\usepackage{chngpage}	% we need this to determine if a figure is on an odd or even page

% captions of tables and images
\usepackage[hang,small,sf]{caption}
\renewcommand{\captionfont}{\sffamily\small}
\renewcommand{\captionlabelfont}{\bfseries}


\usepackage{float}
\usepackage{placeins}
% \floatstyle{ruled}
%\floatplacement

\renewcommand{\floatpagefraction}{0.85}	% if a figure takes more than 85% of a page it will be typeset on a separate page
\usepackage[it,bf,tight,hang,raggedright]{subfigure}

%\numberwithin{figure}{section}
%\numberwithin{table}{section}

% =============================
% = Wide Bibliography Command =
% =============================
\newcommand{\widebibliography}[1]{
  \fancyhfoffset[LE,RO]{\marginparsep+\marginparwidth-4cm}
  \fancyheadoffset[LO,RE]{0cm}
  \fancyfootoffset[RE]{2cm}
  \newgeometry{inner=2cm, outer=2cm, bottom=4cm} 
}



%
%  HEADER
%

% stolen from i10
% Change page headers and footers:
\usepackage{calc}
\usepackage{fancyhdr}
\pagestyle{fancy}
\fancyhfoffset[LE,RO]{\marginparsep+\marginparwidth}
\fancyhfoffset[RE,LO]{2cm}
% \fancyheadoffset[RE,LO]{\hoffset + \oddsidemargin}
\renewcommand{\headrule}{{\color{LightBlue}% 
  \hrule width\headwidth height\headrulewidth \vskip-\headrulewidth}}
\fancyhf{}
\fancyhead[RE]{\slshape \nouppercase{\leftmark}}    % Even page header: "page   chapter"
\fancyhead[LO]{\slshape \nouppercase{\rightmark}}   % Odd  page header: "section   page"
\fancyhead[RO,LE]{\bfseries \thepage}
\fancyfoot[LE]{\STYLEleftpicture}
\fancyfoot[RO]{\STYLErightpicture}
\fancyfoot[RE]{\STYLEfootnotetext}
\renewcommand{\headrulewidth}{1pt}    % Underline headers
\renewcommand{\footrulewidth}{0pt}    

\fancypagestyle{plain}{               % No chapter+section on chapter start pages
\fancyhf{}
% \fancyhead[LO,RE]{}
\fancyhead[RE]{\slshape \nouppercase{\leftmark}}    % Even page header: "page   chapter"
\fancyhead[LO]{\slshape \nouppercase{\rightmark}}   % Odd  page header: "section   page"
\fancyhead[RO,LE]{\bfseries \thepage}
\fancyfoot[LE]{\STYLEleftpicture}
\fancyfoot[RO]{\STYLErightpicture}
\fancyfoot[RE]{\STYLEfootnotetext}
\renewcommand{\headrulewidth}{1pt}
\renewcommand{\footrulewidth}{0pt}
}


% Change Chapter/Section styles

% Replaced behaviour with additional titlesec commands.
% \usepackage{sectsty}
% \allsectionsfont{\color{LightBlue}\rmfamily\normalfont}


\usepackage{titlesec}

\newcommand{\allsectionformat}{\color{LightBlue}\rmfamily\normalfont}

\titleformat{\part}{\Huge}{\thispagestyle{empty}\color{DarkBlue}\rmfamily\normalfont\partname{ }\thepart}{1pc}{\center\Huge\color{DarkBlue}\scshape}
% \titleformat{\part}{\Huge}{\thispagestyle{empty}\color{DarkBlue}\rmfamily\normalfont\partname{ }\thepart}{1pc}{\center\Huge\color{DarkBlue}\rmfamily\normalfont}


% \titleformat*{\section}{\allsectionformat}
% \titleformat*{\subsection}{\allsectionformat}
% \titleformat*{\subsubsection}{\allsectionformat}
% \titleformat*{\paragraph}{\allsectionformat}
% \titleformat*{\subparagraph}{\allsectionformat}
\addtokomafont{section}{\allsectionformat}
\addtokomafont{subsection}{\allsectionformat}
\addtokomafont{subsubsection}{\allsectionformat}
\addtokomafont{paragraph}{\allsectionformat}
\addtokomafont{subparagraph}{\allsectionformat}

\titlespacing*{\section}{0pt}{0pt}{0pt}
\titleformat{\chapter}{\Huge}{\color{DarkBlue}\rmfamily\normalfont\hspace{-1cm}\chaptertitlename{ }\thechapter}{1pc}{\Huge\color{DarkBlue}\rmfamily\normalfont}

% Left headings: "1  INTRODUCTION"
%\renewcommand{\chaptermark}[1]{%
%\markboth{\thechapter\ \ \ \ #1}{}}

% Right headings: "1.1  Basics"
%\renewcommand{\sectionmark}[1]{%
%\markright{\thesection\ \ \ \ #1}{}}

%
% PAGE LAYOUT
%
% \topmargin10mm
% \addtolength{\headheight}{2pt} % To avoid overfull vboxes from fancyhdr
% \footskip10mm % space text bottom <-> date stamp
% 
% \oddsidemargin10mm
% \evensidemargin50mm
% 
% \textwidth100mm
% \marginparsep10mm
% \marginparwidth30mm
% 
% 
% \newlength{\fullwidth} % Width of text plus margin notes
% \setlength{\fullwidth}{\textwidth}
% \addtolength{\fullwidth}{\marginparsep}
% \addtolength{\fullwidth}{\marginparwidth}
% 
% \setlength{\headwidth}{\fullwidth} % Header stretches over margin notes

\usepackage{geometry}
\geometry{inner=4cm, outer=6cm, bottom=4cm}


%
%  T Y P E S E T T I N G   -   T W E A K E S
%

%\lefthyphenmin=3
%\righthyphenmax=3

% instead of sloppy
%\tolerance 1414
%\hbadness 1414
\tolerance 2414
\hbadness 2414
\emergencystretch 1.5em
\hfuzz 0.3pt
\widowpenalty=10000     % Hurenkinde r
\clubpenalty=10000      % Schusterjungen
\brokenpenalty=10000
\interlinepenalty=9000 % seitenumbruch im absatz
\vfuzz \hfuzz
\raggedbottom


%
%     U S E R    D E F I N E D    C O M M A N D S
%

% speacial abstrac environment
\newenvironment{abstract}[1][]%
  {\chapter*{#1}\label{cha:#1}\begin{itshape}\begin{center}}{\end{center}\end{itshape}}

\newenvironment{acknowledgments}[1][Acknowledgments]%
  {\chapter*{#1}\label{cha:#1}}{}


% special compact list
\newlist{scenario}{enumerate}{1}
\setlist[scenario,1]{label=\color{LightBlue}(\alph*),itemsep=2pt, parsep=2pt}

% special variable list
% with longest label
\newenvironment{VarDescription}[1]%
  {\begin{list}{}{\renewcommand{\makelabel}[1]{\color{LightBlue}\rm\itshape{##1}\hfill}%
    \settowidth{\labelwidth}{\color{LightBlue}\rm\itshape{#1}\hfill}%
    \setlength{\leftmargin}{\labelwidth}\addtolength{\leftmargin}{\labelsep}}}%
  {\end{list}}
  
% special variable list with dots
% with longest label
\newenvironment{DotVarDescription}[2][]%
  {\begin{list}{}{\renewcommand{\makelabel}[1]{\color{LightBlue}\rm\itshape{##1}\dotfill{#1}}%
    \settowidth{\labelwidth}{\color{LightBlue}\rm\itshape{#2}\dotfill{#1}}%
    \setlength{\leftmargin}{\labelwidth}\addtolength{\leftmargin}{\labelsep}}}%
  {\end{list}}

%\chapterquote	{ QUOTATION }
%				{ SOURCE }
%outputs a quote with its source, can be used as an introduction to chapters
\newcommand{\chapterquote}[2]{
\begin{quotation}
    \begin{flushright}
	\noindent\emph{``{#1}''\\[1.5ex]---{#2}}
    \end{flushright}
\end{quotation}
}

% Margin notes
\usepackage{mparhack}
\ifdefined\mnotefootnote
  \newcommand{\mnote}[1]{\marginpar{\raggedright\textsf{{\footnotesize{\color{LightBlue}#1}}}}}
\else
  \newcommand{\mnote}[1]{\marginpar{\raggedright\textsf{{\color{LightBlue}#1}}}}
\fi

%----------------------------------------------------------------------------------
% 
% \myFigure	{ FILENAME (without or without extension) }
%			{ LABEL }
%			{ CAPTION TEXT }
%			{ SHORT VERSION OF CAPTION TEXT }
%
%Bild wird in der Breite der textspalte gesetzt
%picture using the width of the text column
\newcommand{\myFigure}[4]%
{%
\begin{figure}[ht!bp]%
	\begin{center}%
		\includegraphics[width= \textwidth]{#1}%
		\caption[#4]{#3}
		\label{#2}%
	\end{center}%
\end{figure}%
}%
%----------------------------------------------------------------------------------
% \myBigFigure	{ FILENAME (without or without extension) }
%				{ LABEL }
%				{ CAPTION TEXT }
%				{ SHORT VERSION OF CAPTION TEXT }
%
%Bild wird in kompletter Breite gesetzt
%picture using full width of the page
\newcommand{\myBigFigure}[4]
{
\begin{figure}[t!bp]
	\checkoddpage
	\ifcpoddpage
		%nothing
	\else
		\hspace{-\marginparsep}\hspace{-\marginparwidth}
	\fi
	%use minipage to center the label beneath the figure
	\begin{minipage}{\fullwidth}
		\includegraphics[width= \fullwidth]{#1}
		\caption[#4]{#3}
		\label{#2}
	\end{minipage}
\end{figure}
}
